 
 %参考《控制理论与应用》提供的LATEX模板  http://jcta.alljournals.ac.cn/uploadfile/cta_cn/20170419/kzllyy%20template20170419-2.9.zip
 % BHOSC   BUAAthesis  https://github.com/BHOSC/BUAAthesis/
 % 北航学报 http://bhxb.buaa.edu.cn/UserFiles/File/%E5%8C%97%E8%88%AA%E5%AD%A6%E6%8A%A5%E6%A8%A1%E6%9D%BF17.1.16(1).doc
 
 %%%% 五号字对应10.5pt,不知道这样设置对否?
\documentclass[10.5pt,twocolumn]{jthu-st}


%%画圆圈数字
\newcommand*\circled[1]{\tikz[baseline=(char.base)]{
            \node[shape=circle,draw,inner sep=1pt] (char) {#1};}}
            
%取消英文连词符
% \tolerance=1
% \emergencystretch=\maxdimen
% \hyphenpenalty=10000
% \hbadness=10000

\newcommand\mycolorRed[1]{{\color{red}#1}}
\newcommand\mycolorYellow[1]{{\color{yellow}#1}}
% \newcommand*\mycolorRed{\color{red}}

%%%????? 公式中字体的定义尺寸为 10 磅,上标/下标 68%,次下标/上标 42% ?????? 
\DeclareMathSizes{10.5}{10}{6.8}{4.2}
%%%% 本示例中带单位的数据采用的是siunitx来生成,好像默认与公式同样大小的字体,所以数字在正文中会小一些
%%%% 行文中普通数字大小为10.5pt,公式里或者用siunitx生成的数字则会是10pt,多少有点不协调。

%%%设置公式前后距离,差不多近似
\setlength{\abovedisplayskip}{2.5mm}
\setlength{\belowdisplayskip}{2.5mm}


\usepackage{tabu}
\usepackage{longtable}
\usepackage{makecell}
\usepackage{placeins}
\renewcommand\cellgape{\Gape[-3pt][-3pt]}


%%%%%%%%%%%%%%%%%%%%%%%%%%%%%%%%%%%%%%%%%%%%%%%%%%%%%%%%%%%%%%%%
%      文章正文
%%%%%%%%%%%%%%%%%%%%%%%%%%%%%%%%%%%%%%%%%%%%%%%%%%%%%%%%%%%%%%%%
\begin{document}
%%%%%%%%%%%%%%%%%%%%%%%%%%%%%%%%%%%%%%%%%%%%%%%%%%%%%%%%%%%%%%%%
% 标题,基金项目,作者,通信地址定义
%%%%%%%%%%%%%%%%%%%%%%%%%%%%%%%%%%%%%%%%%%%%%%%%%%%%%%%%%%%%%%%%
\title{
\erhao{\bf 基于分水岭算法的细胞图像分割及其改进} 
}

\author{ \sihao\kai 
  杨博文 \makebox{$^{\text{1}}$}、汲浩冉 \makebox{$^{\text{2}}$}、张奕驰 \makebox{$^{\text{3}}$}\\
\liuhao (
\liuhao  1.~~北京师范大学~~人工智能学院~~计算机科学与技术专业,202311081015 \\
\liuhao  2.~~北京师范大学~~人工智能学院~~计算机科学与技术专业,202311998246 \\
\liuhao  3.~~北京师范大学~~人工智能学院~~计算机科学与技术(公费师范)专业,202311081002) 
}

\date{}  % 这一行用来去掉默认的日期显示
%%%%%%%%%%%%%%%%%%%%%%%%%%%%%%%%%%%%%%%%%%%%%%%%%%%%%%%%%%%%%%%%
% 奇数页页眉
%%%%%%%%%%%%%%%%%%%%%%%%%%%%%%%%%%%%%%%%%%%%%%%%%%%%%%%%%%%%%%%%
\fancyhead[CO]{{\footnotesize 杨博文、汲浩冉、张奕驰: 基于分水岭算法的细胞图像分割及其改进}}            %请在这里写出第一作者以及论文题目
%%%%%%%%%%%%%%%%%%%%%%%%%%%%%%%%%%%%%%%%%%%%%%%%%%%%%%%%%%%%%%%%
%%%%%%%%%%%%%%%%%%%%%%%%%%%%%%%%%%%%%%%%%%%%%%%%%%%%%%%%%%%%%%%%
%  显示title,并设页码为空(按杂志社要求)
%%%%%%%%%%%%%%%%%%%%%%%%%%%%%%%%%%%%%%%%%%%%%%%%%%%%%%%%%%%%%%%%
%%%%%%%%%%%%%%%%%%%%%%%%%%%%%%%%%%%%%%%%%%%%%%%%%%%%%%%%%%%%%%%%
%      中文摘要
%%%%%%%%%%%%%%%%%%%%%%%%%%%%%%%%%%%%%%%%%%%%%%%%%%%%%%%%%%%%%%%%
\CKeyword{细胞分割;Otsu算法;分水岭算法;U-Net}

\twocolumn[
  \begin{@twocolumnfalse}
  \maketitle
% \positiontextbox{2.2cm}{2.9cm}{\wuhao http://bhxb.buaa.edu.cn \quad  jbuaa@buaa.edu.cn\\[0.3cm]
% \wuhao DOI: \ 10.13700/j.bh.1001-5965.****.****}
\begin{CAbstractJBUAA}
细胞分割是生物医学图像分析中的关键问题之一,其目标是在图像中准确区分并定位单个细胞。本实验以Otsu算法和传统分水岭算法为起点,通过不断对结果的分析进行方法的改进。本实验发现,非细胞因素对于传统Otsu算法、分水岭算法的结果影响较大且该方法不能很好的分割粘连细胞。针对上述问题,本文在传统分水岭算法上不断加以改进,最终提出一种结合U-Net分割前景与背景、形态学分析、基于细胞自适应参数的极大值点作为注水点与分水岭算法进行融合改进。该方法采用U-Net网络对细胞前景进行语义分割,获得细胞概率图。随后对概率图进行二值化与距离变换,基于形态学特征判断是否已经为合适的细胞,通过基于细胞自适应参数的局部极大值检测生成初始注水点。进一步地,将距离变换结果与图像梯度信息进行加权融合,并基于此利用分水岭算法实现细胞实例分割。实验在 Data Science Bowl 2018 细胞数据集上进行,结果表明,所提出的方法在细胞数量估计与实例分割质量方面均取得了较好的效果,相比基础分水岭方法显著降低了计数误差。
\end{CAbstractJBUAA}
%%%%%%%%首页角注
\positiontextbox{2.0cm}{25.5cm}{
\noindent\rule{4cm}{.5pt}\\[0.5ex]%
\hspace*{1em} \liuhao \linespread{0.8}\selectfont
\parbox{\textwidth}{%
{\hei\makebox[\widthof{\makebox{*}收}][r]{\makebox{*}作}者简介:}杨博文,北京师范大学人工智能学院计算机科学与技术专业本科在读,学号202311081015。\\汲浩冉,北京师范大学人工智能学院计算机科学与技术专业本科在读,学号202311998246。\\张奕驰,北京师范大学人工智能学院计算机科学与技术(公费师范)专业本科在读,学号202311081002。
}}
  \end{@twocolumnfalse}
]


%%%%%%%%%%%%%%%%%%%%%%%%%%%%%%%%%%%%%%%%%%%%%%%%%%%%%%%%%%%%%%%%
%  正文由此开始-------------------------
%%%%%%%%%%%%%%%%%%%%%%%%%%%%%%%%%%%%%%%%%%%%%%%%%%%%%%%%%%%%%%%%
%%%%%%%%%%%%%%%%%%%%%%%%%%%%%%%%%%%%%%%%%%%%%%%%%%%%%%%%%%%%%%%%
\wuhao 
%  分栏开始

%%%%%!!!!!正文在第一页两栏分别合适位置插入 \enlargethispage{-3.3cm},给首页跨双栏脚注留空间,大小需要结合前面位置和高度手动设置!!!!!
%%%%%%%%%%%%%%%%%%%%%%%%%%%%%%%%%%%%%%%%%%%%%%%%%%%%%%%%%%%%%%%%

\enlargethispage{-2.1cm}% 这个是干啥的我也没太懂,反正大概能空行

\section{项目背景与意义} %本section由zyc修改

应当包括:所选问题讲解,背景,意义% zyc

\enlargethispage{-2.1cm}% 这个是干啥的我也没太懂,反正大概能空行

\section{实验数据说明} %本section由zyc修改

应当包括:数据集说明,数据集使用说明(只用stage1\_train,共670张图片,U-Net部分如何分割),四张典型图的选择说明(分别是:分割、车轮、光、甜甜圈),包括原图、GT-Mask图(答案)% zyc

\section{模型原理}

\subsection{Otsu算法讲解} %jhr
% jhr

\subsection{分水岭算法讲解} %jhr
分水岭算法是一种基于拓扑学的图像分割算法,主要应用于将图像分割为多个具有连通性或相似性的区域,尤其适用于从背景中提取近乎一致的物体。对于分水岭算法,我们常常将灰度图像视为一幅地形图,其中灰度值则表示地形的高度:灰度值较小的区域我们认为是“低洼地带”,而灰度值较大的区域则对应“山脊”。
我们从“低洼地带”开始注水(即灰度值较小的区域),水位会不断地上升,当水位上升至一定高度时,不同的水源会汇合,此时,我们需要在汇合处建立“堤坝”作为两股水源的分界线。随着水位的继续上升,更多的“堤坝”会被建立起来,直到整个图像都被水淹没为止。最终,这些“堤坝”就形成了图像的分割边界。% jhr

\subsection{U-Net模型讲解} %ybw
U-Net是一种针对生物医学图像分割任务的卷积神经网络架构,由Olaf Ronneberger等人于2015年提出\textsuperscript{\cite{unet}}。如图~\ref{fig:unet}所示,该模型由两部分组成:一个收缩路径用于捕获上下文信息,和一个对称的扩展路径用于精确的局部化。这种结构使得网络能够在有限的训练样本下,提供高精度的分割结果。U-Net的关键设计思想是通过上采样操作恢复高分辨率特征,并与收缩路径中的相应特征图进行拼接,从而提高分割的精确度。

\begin{figure}[htbp]
\centering
\includegraphics[width=\linewidth,trim=0 0 0 0]{./image/unet模型.png}
\bicaption[fig:unet]{图}{\centering U-Net模型原理示意图}{Fig.}{\centering Schematic Diagram of the U-Net Model}
\end{figure}

U-Net的网络结构没有全连接层,而是通过卷积层逐步提升图像的分辨率。在训练过程中,U-Net采用了数据增强技术,这使得网络能够更好地学习到图像中的变形特征,增强其鲁棒性和泛化能力。该架构被广泛应用于多种生物医学图像分割任务,优势在于即使在训练数据较少的情况下,依然能够提供高质量的分割效果,并且具有较快的处理速度。

% ybw

\section{项目主要工作与解决方法}

\subsection{实验总流程} 
%这一部分由jhr写,合理分段,subsection代表一、1.这个级别
本实验将以传统的阈值分割算法——Otsu算法为起点,结合分水岭算法和深度学习中的U-Net模型,逐步优化细胞分割的效果。
其中,分水岭算法将采用基于距离变化和基于局部极大值的注水点选取方法,分别观察细胞分割效果的提升情况,并尝试引入梯度进行进一步的效果优化。
最后,我们将引入U-Net模型进行前后景分割,结合前述的分水岭算法,进一步提升细胞分割的准确性,在这一部分中,我们将分别采取形态学,固定/可变阈值以及梯度的手段来分别优化分水岭算法的效果。
具体的实验流程如图~\ref{fig:experimentflow}所示。
\begin{figure}[htbp]
\centering
\includegraphics[width=\linewidth,trim=0 0 0 0]{./image/流程图.png}
\bicaption[fig:experimentflow]{图}{\centering 实验总流程图}{Fig.}{\centering Schematic Diagram of the Experimental Flow}
\end{figure}

在本次实验中,我们将通过视觉与数值两种方式对各个方法的分割效果进行评估。视觉评估主要通过观察分割结果图像(主要观察是否识别出所有细胞、细胞粘连问题是否解决以及是否存在过分割问题),视觉效果将采用如图~\ref{fig:example}的样本图进行效果观察;数值评估则通过计算验证集中所有图片细胞分割数量的相对误差(基准细胞分割数量由kaggle平台提供)的平均值来评估,数值越小,则说明分割效果越好。

\begin{figure}[htbp]
\centering
\includegraphics[width=\linewidth,trim=0 0 0 0]{./image/原图.png}
\bicaption[fig:example]{图}{\centering 样本图}{Fig.}{\centering Example of Sample Image}
\end{figure}

\subsection{使用Otsu算法进行细胞分割} 
%这一部分由jhr写,合理分段,subsection代表一、1.这个级别
如前文所述,Otsu算法是一种经典的基于图像灰度直方图的自动阈值分割方法。我们首先尝试使用Otsu算法对细胞图像进行二值化处理,以实现初步的细胞分割。实现步骤也较为简单,我们首先利用Otsu算法计算出最佳阈值(可调库),然后根据阈值进行二值化处理(即前景与背景的划分),最后对二值化结果进行连通域分析,统计细胞数量。分割结果如图~\ref{fig:基于Otsu算法的阈值分割结果}所示。
\begin{figure}[htbp]
\centering
\includegraphics[width=\linewidth,trim=0 0 0 0]{./image/otsu.png}
\bicaption[fig:基于Otsu算法的阈值分割结果]{图}{\centering 基于Otsu算法的阈值分割结果}{Fig.}{\centering The Result of Otsu Algorithm-based Threshold Segmentation}
\end{figure}

首先,该算法细胞分割数量的平均相对误差为0.2920,可以发现效果并不理想。我们再观察图~\ref{fig:基于Otsu算法的阈值分割结果},可以发现该方法在处理细胞粘连问题上表现较差,许多粘连的细胞被错误地识别为单个细胞,导致分割数量偏少。并且,我们发现当噪声较大,或者细胞粘连十分严重时,Otsu算法无法有效地分割细胞。

\subsection{使用基于距离变换的分水岭算法进行细胞分割}
观察发现,基于Otsu算法的细胞分割效果并不理想,存在着细胞粘连无法分割等问题。而我们知道,分水岭算法在处理图像分割问题上具有较好的效果,尤其是在处理粘连物体的分割上表现突出。因此,我们尝试引入分水岭算法来改进细胞分割效果。需要注意的是,对于分水岭算法来说,预处理过程尤其重要,主要分为两个部分:二值化过程——用于确定前景和背景;注水点选取——选取细胞的核心区域。在这里,我们主要考虑如何选取注水点,二值化过程我们仍然采用Otsu算法进行处理。注水点的选取我们将采取基于距离变换和基于局部极大值两种方法,分别观察其分割效果。

首先,我们介绍基于距离变换的注水点选取方法。具体来说,我们首先对二值化结果进行距离变换,得到每个前景像素点到最近背景像素点的距离值。然后,我们通过设定一个阈值——前景到背景最大的距离的某个比例,来选取注水点。我们认为,当距离值大于设定阈值时,则像素点距离背景边界越远,则越可能属于细胞的核心区域,因此我们将这些点作为注水点。接着,我们从这些注水点开始不断地向外膨胀,直到遇到背景或两个膨胀区域相遇为止,建立分水岭线,完成细胞的分割。分割结果如图~\ref{fig:基于距离变换的分水岭算法结果}所示。

\begin{figure}[htbp]
\centering
\includegraphics[width=\linewidth,trim=0 0 0 0]{./image/基于距离变换的分水岭算法.png}
\bicaption[fig:基于距离变换的分水岭算法结果]{图}{\centering 基于距离变换的分水岭结果}{Fig.}{\centering The Result of Distance Transform-based Watershed Segmentation}
\end{figure}

该算法的细胞分割数量的平均相对误差为0.2644,相较于单纯的Otsu算法有了显著提升。观察图~\ref{fig:基于距离变换的分水岭算法结果},可以发现该算法在处理细胞粘连问题上有了一定的改善,相较于单纯的Otsu算法,可以对一些粘连细胞进行分割,但是仍然存在大部分粘连细胞无法分割的问题。此外,当噪声较大,或者细胞粘连十分严重时,同Otsu算法一样,无法有效地分割细胞。

\subsection{使用基于局部极大值的分水岭算法进行细胞分割}
接下来,我们介绍基于局部极大值的注水点选取方法。由于基于距离变换的方法的距离阈值是基于全局信息设定的,且是固定的。那么,当设定的阈值过大时,可能会出现某些距离背景过近的细胞核心区域无法被选取为注水点,导致这些细胞无法被分割;而当设定的阈值过小时,则会出现粘连细胞的注水点过于接近从而识别为同一个连通区域,无法进行粘连细胞的有效分割。为了解决上述问题,我们引入了自适应的基于局部极大值的注水点选取方法。

具体来说,我们首先对二值化结果进行距离变换。然后,我们考虑每个前景像素点与其周围前景像素点的距离值进行比较,若该像素点的距离值大于其所有周围像素点的距离值,则认为该像素点为局部极大值点,即细胞的核心区域。接着,我们从这些注水点开始不断地向外膨胀,直到遇到背景或两个膨胀区域相遇为止,建立分水岭线,完成细胞的分割。分割结果如图~\ref{fig:基于局部极大值的分水岭结果}所示。
\begin{figure}[htbp]
\centering
\includegraphics[width=\linewidth,trim=0 0 0 0]{./image/基于局部极大值的分水岭.png}
\bicaption[fig:基于局部极大值的分水岭结果]{图}{\centering 基于局部极大值的分水岭结果}{Fig.}{\centering The Result of Local Maxima-based Watershed Segmentation}
\end{figure}

该算法的细胞分割数量的平均相对误差为0.2318,相较于基于距离变换的分水岭算法有了进一步提升。观察图~\ref{fig:基于局部极大值的分水岭结果},可以发现该算法在处理细胞粘连问题上表现更好,能够有效地分割大部分粘连细胞。但是,仍然存在小部分粘连细胞无法分割的问题。并且,该算法有了新的问题——过分割问题,会发现有些单个细胞被错误地分割成多个部分,导致分割数量偏多。同时,当噪声较大或粘连细胞过于严重时,仍然无法有效地分割细胞。

\subsection{使用梯度优化分水岭算法进行细胞分割}
基于教材的说法,分水岭分割的主要应用之一是从背景中提取近乎一致的物体。由变化较小的灰度表征的区域有较小的梯度值。因此,我们经常见到分水岭分割方法用于一幅图像的梯度,而不是图像本身。为此,我们尝试在前述的两种分水岭算法的基础上,引入梯度信息来优化分割效果。

我们首先对梯度图像进行归一化处理,然后将归一化后的梯度图像与二值化图像进行加权融合,得到新的图像。接着,我们分别在基于距离变换和基于局部极大值的分水岭算法中,使用该融合图像进行注水点选取和分割。分割结果如图~\ref{fig:基于距离变换的梯度优化分水岭结果}和图~\ref{fig:基于局部极大值的梯度优化分水岭结果}所示。
\begin{figure}[htbp]
\centering
\includegraphics[width=\linewidth,trim=0 0 0 0]{./image/基于距离变换的梯度优化分水岭结果.png}
\bicaption[fig:基于距离变换的梯度优化分水岭结果]{图}{\centering 基于距离变换的梯度优化分水岭结果}{Fig.}{\centering The Result of Gradient-optimized Distance Transform-based Watershed Segmentation}
\end{figure}

\begin{figure}[htbp]
\centering
\includegraphics[width=\linewidth,trim=0 0 0 0]{./image/基于局部极大值的梯度优化分水岭结果.png}
\bicaption[fig:基于局部极大值的梯度优化分水岭结果]{图}{\centering 基于局部极大值的梯度优化分水岭结果}{Fig.}{\centering The Result of Gradient-optimized Local Maxima-based Watershed Segmentation}
\end{figure}

基于距离变换的梯度优化分水岭结果和基于局部极大值的梯度优化分水岭结果的细胞分割数量的平均相对误差分别为0.2644和0.2318。观察图~\ref{fig:基于局部极大值的梯度优化分水岭结果}和图~\ref{fig:基于局部极大值的分水岭结果},可以发现引入梯度信息后,算法并没有进一步得到改善。
% jhr

\subsection{使用U-Net进行前后景分割}

如上文所言,我们在使用简单二值化方法时发现,二值化方法具有阈值不好确定、对于特殊图像效果极差的问题,所以我们考虑引入常用于医学图像分割的U-Net进行前后景分割。

由于这里我们只是引入U-Net卷积神经网络作为一个小小的步骤,而不是整个流程的核心,所以也不算是违背了最好不要使用神经网络简单训练一个模型的要求。

具体来讲,对于训练集,我们以其原图像和结果细胞图(原数据集中每个细胞一张图,这里将其合并为一张图)作为样本和标签,使U-Net具有能够区分每个点是前景(细胞)的概率,然后根据概率给定阈值,高于阈值的则认为其为细胞的组成部分,反之则认为其为背景。

首先,我们测试仅改变二值化方法(简单二值化、U-Net二值化),即U-Net+自适应分水岭方法+对原图使用分水岭。依然考虑我们四张典型图片,分割结果如图~\ref{fig:unet+zi+yuan:totalresult}。

\begin{figure}[htbp]
\centering
\includegraphics[width=\linewidth,trim=0 0 0 0]{./image/U-Net+自适应分水岭+原图总结果.png}
\bicaption[fig:unet+zi+yuan:totalresult]{图}{\centering 改进的基于U-Net做前景标记的自适应分水岭结果}{Fig.}{\centering The Result of Improved Adaptive Watershed with U-Net Foreground Markers}
\end{figure}

首先,从分割细胞数量平均相对误差来看,其结果相较于改进的自适应分水岭(基于原图)结果,在其他变量均不变的情况下,从0.2318提升降低到了0.1746。这是一个十分显著的提升。观察结果图,我们可以看到,其分割结果好了许多,尤其是第三、四图。第三图的左下角部分的光线没有被识别为细胞进行分割。第四图没有被完全整体识别为一个细胞,其环形结构被识别,部分边界被分开。第一、二图的结果也比之前的要更好,一部分连结细胞被分开。但是仍然存一些欠分割问题。

\subsection{使用变值方法基于细胞大小进行注水点选取}

如上文所言,我们的自适应分水岭算法是根据极大值点确定注水点的,其中有一定的要求,比如说min\_distance代表如果两个极大值点的差距小于min\_distance的话就不将其视为两个。那么这里我们发现,这对于不同细胞的适应能力是较差的。有的细胞比较大,那这个数值就应当相应的提高,反之则应当相应变小。所以接下来我们使用变值方法决定min\_distance。

这我们定义,min\_distance是0.8倍的最大距离(后续0.8会进行修改,在最后进行消融实验)。实验结果如图~\ref{fig:unet+zi+bian+yuan:totalresult}。

\begin{figure}[htbp]
\centering
\includegraphics[width=\linewidth,trim=0 0 0 0]{./image/U-Net+自适应分水岭+变值+原图总结果.png}
\bicaption[fig:unet+zi+bian+yuan:totalresult]{图}{\centering 改进的基于U-Net做前景标记的使用变阈值的自适应分水岭结果}{Fig.}{\centering The Result of U-Net-based Adaptive Watershed with Adaptive Thresholds}
\end{figure}

从分割细胞数量平均相对误差来看,其达到0.1721,相比前文有所增长但幅度不大。从结果图来看,其对第二图的结果进行了尤其的改变,分割更多了,解决了一定的欠分割问题。与此同时,其导致了过分割与欠分割并存的问题。

\subsection{使用梯度作为分水岭图像}

我们再来尝试梯度图像。在本章前半部分,在基于二值化的分割方法之中,我们发现梯度对于结果的影响几乎没有。这是真的吗?如果是这样的话为什么教材中会写一般使用梯度图像进行分水岭呢?于是在我们使用U-Net做前景标记之后,我们决定使用梯度图像看一下效果。结果如图~\ref{fig:unet+zi+bian+grad:totalresult}。

\begin{figure}[htbp]
\centering
\includegraphics[width=\linewidth,trim=0 0 0 0]{./image/U-Net+自适应分水岭+变值+梯度总结果.png}
\bicaption[fig:unet+zi+bian+grad:totalresult]{图}{\centering 改进的基于U-Net做前景标记的使用变阈值的使用梯度图像的自适应分水岭结果}{Fig.}{\centering The Result of U-Net-based Adaptive Watershed with Adaptive Thresholds and Gradient Images}
\end{figure}

从分割细胞数量平均相对误差来看,其达到0.1346,相较前文有较大的提升。从结果图上来看,其更清晰的关注到了边界,开始关注内部的线条。但是,其对于概率图变色过于敏感可能导致结果不够好,如第四图出现从中间直接连到边界的不完全分割情况。

\subsection{使用梯度-原始概率图加权图像作为分水岭图像}
既然梯度单独不行,那么我们同时考虑梯度和原始概率图,将二者加权以期待更好的结果,这里的权重为梯度图$0.01$、原始概率图$0.99$。结果如图~\ref{fig:unet+zi+bian+quan:totalresult}。

\begin{figure}[htbp]
\centering
\includegraphics[width=\linewidth,trim=0 0 0 0]{./image/U-Net+自适应分水岭+变值+加权总结果.png}
\bicaption[fig:unet+zi+bian+quan:totalresult]{图}{\centering 改进的基于U-Net做前景标记的使用变阈值的使用梯度-概率加权图像的自适应分水岭结果}{Fig.}{\centering The Result of U-Net-based Adaptive Watershed Using Gradient–Probability Weighted Images}
\end{figure}

从分割细胞数量平均相对误差来看,其达到0.1323,相较前文又有一定的提升。从结果图上来看,其改进了从中间直接连到边界的不完全分割情况,分割结果越来越好。但是,注意第一图中下部分,出现“甜甜圈”现象。即,对于细胞和细胞核,呈现从细胞核到细胞边缘被切成多份的情况。

\subsection{使用形态学方法改善“甜甜圈”问题}

我们引入形态学内容来改进“甜甜圈”现象。当一个分割结果形状接近圆、椭圆且离心率较小的时候,我们认为这已经是一个分割好的细胞了不再需要进行分割。其结果如图~\ref{fig:unet+zi+bian+xing+quan:totalresult}。

\begin{figure}[htbp]
\centering
\includegraphics[width=\linewidth,trim=0 0 0 0]{./image/U-Net+自适应分水岭+变值+形态+加权总结果.png}
\bicaption[fig:unet+zi+bian+xing+quan:totalresult]{图}{\centering 改进的基于U-Net做前景标记的加入形态学判断的使用变阈值的使用梯度-概率加权图像的自适应分水岭结果}{Fig.}{\centering The Result of U-Net-based Adaptive Watershed with Morphology and Weighted Images}
\end{figure}

从分割细胞数量平均相对误差来看,其变差了一点,变为0.1334。但是从结果图上来看,其变好了一些,第一图的“甜甜圈”现象被正确解决。详细分析见本文第5章。% ybw

\section{实验结果展示及分析}

在本章中,我们将聚焦于\textbf{改进的基于U-Net做前景标记的加入形态学判断的使用变阈值的使用梯度-概率加权图像的自适应分水岭}结果,进行一定的分析,并且给出后续可能的改进方向。

\subsection{实验结果展示及分析}

首先,我们连续展示六张结果图。这些图都来自于改进的基于U-Net做前景标记的使用变阈值的使用梯度-概率加权图像的自适应分水岭结果,包括四个子图,从左到右依次是原图、GT Mask(答案图)、对哪张图做分水岭(加权图)、划分结果。

\begin{figure}[htbp]
\centering
\includegraphics[width=\linewidth,trim=0 0 0 0]{./image/result1.png}
\bicaption[fig:result1]{图}{\centering 改进的基于U-Net做前景标记的使用变阈值的使用梯度-概率加权图像的自适应分水岭结果1}{Fig.}{\centering The Result1 of U-Net-based Adaptive Watershed Using Gradient–Probability Weighted Images}
\end{figure}

\begin{figure}[htbp]
\centering
\includegraphics[width=\linewidth,trim=0 0 0 0]{./image/result2.png}
\bicaption[fig:result2]{图}{\centering 改进的基于U-Net做前景标记的使用变阈值的使用梯度-概率加权图像的自适应分水岭结果2}{Fig.}{\centering The Result2 of U-Net-based Adaptive Watershed Using Gradient–Probability Weighted Images}
\end{figure}

\begin{figure}[htbp]
\centering
\includegraphics[width=\linewidth,trim=0 0 0 0]{./image/result3.png}
\bicaption[fig:result3]{图}{\centering 改进的基于U-Net做前景标记的使用变阈值的使用梯度-概率加权图像的自适应分水岭结果3}{Fig.}{\centering The Result3 of U-Net-based Adaptive Watershed Using Gradient–Probability Weighted Images}
\end{figure}

\begin{figure}[htbp]
\centering
\includegraphics[width=\linewidth,trim=0 0 0 0]{./image/result4.png}
\bicaption[fig:result4]{图}{\centering 改进的基于U-Net做前景标记的使用变阈值的使用梯度-概率加权图像的自适应分水岭结果4}{Fig.}{\centering The Result4 of U-Net-based Adaptive Watershed Using Gradient–Probability Weighted Images}
\end{figure}

\begin{figure}[htbp]
\centering
\includegraphics[width=\linewidth,trim=0 0 0 0]{./image/result5.png}
\bicaption[fig:result5]{图}{\centering 改进的基于U-Net做前景标记的使用变阈值的使用梯度-概率加权图像的自适应分水岭结果5}{Fig.}{\centering The Result5 of U-Net-based Adaptive Watershed Using Gradient–Probability Weighted Images}
\end{figure}

\begin{figure}[htbp]
\centering
\includegraphics[width=\linewidth,trim=0 0 0 0]{./image/result6.png}
\bicaption[fig:result6]{图}{\centering 改进的基于U-Net做前景标记的使用变阈值的使用梯度-概率加权图像的自适应分水岭结果6}{Fig.}{\centering The Result6 of U-Net-based Adaptive Watershed Using Gradient–Probability Weighted Images}
\end{figure}

首先,我们来看图~\ref{result4},这张图展示了在细胞粘连较为简单的情况下,我们的算法可以完美的做出正确分割。这也是我们最为期望的,希望所有图片都像这张图一样完美。

接着,我们来看图~\ref{result5},这张图展示了在细胞粘连较为复杂的情况下,我们的算法基本可以做出正确分割,但对于过于粘连的情况仍然无法进行很有效的切割。

对应的,我们来看图~\ref{result3},这张图展示了在细胞粘连较为复杂的情况下,可能出现虽然我们分割的细胞数量正确,但是是由于同时存在欠分割和过分割导致的,这也就说明了在本问题中,正如第4.1节所说,细胞数量平均相对误差并不能作为唯一的判断结果是否优秀的标准,还需要参考视觉结果。对于这张图而言,我们的结果只在右上角一个地方出现了一组过分割-欠分割。并且,从这张图可以看出,我们的算法能够察觉到一些人眼很难看出的细胞并予以分割出来,这是非常优秀的。

然后,我们来看图~\ref{result6},这张图的右下角我们会发现两个在图像边缘上的特别小的细胞没有被正确分割出来,也就是说我们的算法对于只存在很小一部分的细胞的识别能力是较差的,对于边界情况的处理是不够好的。

最后,我们回到图~\ref{result1}和图~\ref{result2},这两张图向我们展示了对于非常复杂的原始图像,我们的算法存在一定的欠分割问题或者分割不正确问题,但是对于绝大多数的细胞能够做出正确的分割。这是难能可贵的,对于剩下的一部分小误差,可以通过人工进行一定的调节。

\subsection{可能的改进方向}

首先,我们注意到U-Net做前景标记的一个问题,命名为“空心现象”。空心现象指的是当U-Net的前景标记结果中,可能出现一个(或多个)完整的连续细胞中间出现空洞(即不被U-Net认为是细胞)的情况,如图~\ref{fig:空心}所示,在Binary子图的上方中心,有一个明显的空心现象,在其他的细胞中也存在一些。

\begin{figure}[htbp]
\centering
\includegraphics[width=\linewidth,trim=0 0 0 0]{./image/空心.png}
\bicaption[fig:空心]{图}{\centering 空心现象示例图}{Fig.}{\centering Hollow Artifact Example}
\end{figure}

这些空心现象导致,在做分水岭时,中间部分被认为是背景,从而被割裂开。这是U-Net的最大问题。如果能够解决这个问题,我们相信结果会更上一个层次。

我们试图回到传统的二值化方法进行前后景区分,即改进的加入形态学判断的使用变阈值的使用梯度-概率加权图像的自适应分水岭。其结果如图~\ref{fig:binary+zi+bian+xing+quan:totalresult}。

\begin{figure}[htbp]
\centering
\includegraphics[width=\linewidth,trim=0 0 0 0]{./image/改进+变+形+权.png}
\bicaption[fig:binary+zi+bian+xing+quan:totalresult]{图}{\centering 改进的加入形态学判断的使用变阈值的使用梯度-概率加权图像的自适应分水岭结果}{Fig.}{\centering The Result of Improved Adaptive Watershed with Morphology and Weighted Images}
\end{figure}

其效果明显变差,数值结果为0.1678,对于第一、三、四图,都有着极大的问题。这表明相较于U-Net的空心问题,二值化受原始图片影响更加严重。

所以我们想到使用U-Net+二值化的混合方法,我们既需要考虑二值化结果,也需要考虑U-Net结果,二者共同进行前后景分类,从而达到更好的效果。由于时间关系我们只尝试了如果U-Net的前景占比与二值化的前景占比差距过大,则将二值化的前后景反转(这是因为对于部分图片,其本身底色为白色,二值化没有针对这些进行特别的考虑)。该方法的数值结果为0.1861(更差了一些),这里不再放图片。但我们依然认为这是一条可行的道路,比如考虑让二者投票或者利用二值化结果/U-Net结果确定对方的前后景阈值。

之所以认为这是一个可行的改进方向,因为对于图~\ref{fig:mix},其他的所有方法得到的分割结果都不超过10,只有我们尝试的这个二值化+U-Net将其进行了一定的分割。所以我们认为这种方法是具有可取之处的。

\begin{figure}[htbp]
\centering
\includegraphics[width=\linewidth,trim=0 0 0 0]{./image/mix.png}
\bicaption[fig:mix]{图}{\centering 改进的基于U-Net及二值化方法混合做前景标记的加入形态学判断的使用变阈值的使用梯度-概率加权图像的自适应分水岭结果}{Fig.}{\centering The Result of Hybrid Watershed (U-Net + Binarization) with Morphology and Weighted Images}
\end{figure}% ybw

\section{消融实验}

实际上,本次实验过程为不断累加模块的过程,每个模块的包含与不包含均在第四章有完整的阐述。逆向看整个实验的过程即为消融实验的过程。因此,本章中不再进行消融实验的重复阐述,仅说明超参数选取的相关内容。

\subsection{传统分水岭算法参数选取} 
在传统分水岭算法中,参数的选取对于分割效果有着重要影响。我们在前面说到,当距离阈值的选取不合适时,可能会导致某些细胞无法被识别,或者出现细胞粘连无法分割的问题。而我们的距离阈值选取主要依赖于距离变换结果的最大值的某个比例。因此,我们尝试调整该比例参数,以观察其对分割效果的影响。在本次实验中,我们选取了0.1、0.2、0.3、0.4、0.5、0.6六个不同的比例参数,分别进行细胞分割实验,并计算其细胞分割数量的平均相对误差。实验结果如表~\ref{tab:labelTab传统分水岭算法参数选取结果}所示。
\begin{table}[htbp]
\centering 
\captionnamefont{\xiaowuhao\bf}
\captiontitlefont{\xiaowuhao\bf}
\bicaption[tab:labelTab传统分水岭算法参数选取结果]{表}{\centering 传统分水岭算法参数选取结果}{Table}{\centering The Result of Traditional Watershed Algorithm Parameter Selection}
\renewcommand\tabcolsep{1em}
\xiaowuhao 
\begin{tabular}{cc}
\toprule
{距离阈值比例} & {平均相对误差} \\ 
\midrule
0.1 & 0.2742 \\
0.2 & 0.2644 \\
0.3 & 0.2691 \\
0.4 & 0.2778 \\
0.5 & 0.3490 \\
0.6 & 0.4609 \\
\bottomrule
\end{tabular}
\end{table}

可以发现,当距离阈值比例设定为0.2时,分割效果最佳,平均相对误差最低,为0.2644。因此,在后续的传统分水岭算法实验中,我们均采用该参数进行细胞分割。


% jhr

\subsection{U-Net算法参数选取} 
%这一部分由ybw写,合理分段,subsection代表一、1.这个级别
在引入U-Net的算法中,我们使用了基于局部极大值的分水岭算法进行了注水点的选取。而在这个过程中,我们同样设定一个阈值——两个注水点之间的最小距离,以避免选取过于接近的注水点,导致过分割问题。同样,这个阈值是依赖于局部区域的距离最大值乘以一个比例系数来确定的。在实验过程中,我们尝试了0.5、0.6、0.7、0.8、0.9、0.95六个不同的比例系数,分别进行细胞分割实验,并计算其细胞分割数量的平均相对误差。实验结果如表~\ref{tab:labelTabU-Net算法参数选取结果}所示。
\begin{table}[H]
\centering
\captionnamefont{\xiaowuhao\bf}
\captiontitlefont{\xiaowuhao\bf}
\bicaption[tab:labelTabU-Net算法参数选取结果]{表}{U-Net算法参数选取结果}{Table}{The Result of U-Net Algorithm Parameter Selection}
\renewcommand\tabcolsep{1em}
\xiaowuhao
\begin{tabular}{cc}
\toprule
{最小距离比例} & {平均相对误差} \\
\midrule
0.5 & 0.1340 \\
0.6 & 0.1335 \\
0.7 & 0.1335 \\
0.8 & 0.1337 \\
0.9 & 0.1334 \\
0.95 & 0.1335 \\
\bottomrule
\end{tabular}
\end{table}

可以发现,当最小距离比例设定为0.9时,分割效果最佳,平均相对误差最低,为0.1334。因此,在后续的U-Net算法实验中,我们均采用该参数进行细胞分割。% jhr

\section{总结} %本section由zyc修改

\subsection{研究总结}
研究初期,我们采用经典的Otsu自适应阈值分割算法作为基础方法。但是在实验中我们发现了他的局限性:Otsu算法仅能进行全局阈值分割,无法区分相互粘连的细胞个体。即当细胞密度较高或培养条件导致细胞紧密接触时,Otsu算法将整个粘连区域识别为单一前景目标,这一缺陷严重影响了细胞计数和形态分析的准确性。

为解决细胞粘连问题,我们引入了分水岭算法这一经典的形态学分割方法,并且在实验中进行分水岭算法的改进。分水岭算法的引入使分割性能得改善,但是也出现了新的问题:对于噪声干扰较大或细胞边界模糊的图像,分水岭算法易产生严重的过分割;而在细胞极度粘连的情况下,算法仍无法准确识别所有细胞个体。

为进一步提升分割精度,特别是改善粘连处理能力,我们将深度学习框架U-Net与改进分水岭算法融合。在这个过程中,我们尝试了形态学、变参数、加权梯度等方法来进行改进。最终在量化指标上实现了从0.2920到0.1334的显著提升。

\subsection{算法的适用条件}
本研究基于2018 Data Science Bowl数据集进行算法开发与优化,在此数据集上的系统性实验表明,本文提出的方法在特定条件下展现出卓越的分割性能。

在高质量图像的分割方面,对于该数据集中图像质量优异、细胞形态清晰、对比度高的样本,本方法能够实现接近完美的分割效果。如图~\ref{fig:result7},在选定的优质测试图像中,本方法的分割结果与基准标注基本一致。
\begin{figure}[htbp]
\centering
\includegraphics[width=\linewidth,trim=0 0 0 0]{./image/结果用图-优质图像.png}
\bicaption[fig:result7]{图}{\centering 改进的基于U-Net做前景标记的使用变阈值的使用梯度-概率加权图像的自适应分水岭结果7}{Fig.}{\centering The Result7 of U-Net-based Adaptive Watershed Using Gradient–Probability Weighted Images}
\end{figure}

值得注意的是,在实际的细胞成像过程中,受限于实验条件、染色技术、显微镜性能等因素,并非所有图像都能达到理想质量。本研究针对这类低质量图像具有一定的分割能力,并且可以看到一些人眼看不到的细胞如图~\ref{fig:result8},这说明我们的算法在较复杂的环境中依然可以维持一定的性能。

\begin{figure}[htbp]
\centering
\includegraphics[width=\linewidth,trim=0 0 0 0]{./image/肉眼不可见细胞.png}
\bicaption[fig:result8]{图}{\centering 改进的基于U-Net做前景标记的使用变阈值的使用梯度-概率加权图像的自适应分水岭结果8}{Fig.}{\centering The Result8 of U-Net-based Adaptive Watershed Using Gradient–Probability Weighted Images}
\end{figure}

由于我们只使用了2018 Data Science Bowl数据集进行算法开发与优化,没有针对其他数据集进行测试,所以不能保证该算法在其他数据集上的适用性。在实际使用过程中,应当结合数据的自身特点对本算法进行调整、测试,以求得更高的性能。


\subsection{改进方向与未来工作}
尽管U-Net的引入显著提升了细胞分割的整体性能,但我们在实验过程中发现了一个新的问题,即空心现象,如图~\ref{fig:空心}。在某些情况下,特别是对于较大或形态不规则的细胞,U-Net输出的分割结果会在细胞内部产生空洞或不连续区域。后续可能可以进一步通过将U-Net与二值化进行结合来改进此问题。更具体的论述详见第5.2节。

另外,本次实验中部分参数选取基于经验值,后续可以进一步考察不同的参数对实验结果的改善。

在生物学实验中,数细胞往往由人工进行。可以说,本实验的结果结合少量人工对个别图像进行再判断将会节省很大的时间以及精力,在实际使用中提供很大的便利。希望在后续改进中,建立更加科学严谨的实验设计和方法评估体系,并不断改进算法,推动该细胞图像分割技术更加实用。
% zyc

%%%%%%%%%%%%%%%%%%%%%%%%%%%%%%%%%%%%%%%%%%%%%%%%%%%%%%%%%%%%%%%%
%  参考文献
%%%%%%%%%%%%%%%%%%%%%%%%%%%%%%%%%%%%%%%%%%%%%%%%%%%%%%%%%%%%%%%%


\renewcommand\refname{\hei\wuhao\centerline{参考文献(References)}\global\def\refname{参考文献}}
\vskip 12pt

\begin{thebibliography}{10}

\bibitem{otsu}
OTSU N. A threshold selection method from gray-level histograms[J].
IEEE Transactions on Systems, Man, and Cybernetics, 1979, 9(1): 62--66.
doi: \texttt{10.1109/TSMC.1979.4310076}.

\bibitem{watershed}
BEUCHER S, LANTU\'EJOUL C.
Use of watersheds in contour detection[C]//
Workshop on Image Processing, Real-Time Edge and Motion Detection.
1979.

\bibitem{unet}
Ronneberger, O., Fischer, P., \& Brox, T. (2015). U-Net: Convolutional Networks for Biomedical Image Segmentation.
MICCAI 2015, 234-241. Springer, Cham.
doi: \texttt{10.1007/978-3-319-24574-4\_28}

% 如果有换页问题,使用\newpage,这一段由zyc更改,上面这个是我需要引用的参考文献
\end{thebibliography}

\end{document}