\subsection{分水岭算法讲解} %jhr
分水岭算法是一种基于拓扑学的图像分割算法,主要应用于将图像分割为多个具有连通性或相似性的区域,尤其适用于从背景中提取近乎一致的物体。对于分水岭算法,我们常常将灰度图像视为一幅地形图,其中灰度值则表示地形的高度:灰度值较小的区域我们认为是“低洼地带”,而灰度值较大的区域则对应“山脊”。
我们从“低洼地带”开始注水(即灰度值较小的区域),水位会不断地上升,当水位上升至一定高度时,不同的水源会汇合,此时,我们需要在汇合处建立“堤坝”作为两股水源的分界线。随着水位的继续上升,更多的“堤坝”会被建立起来,直到整个图像都被水淹没为止。最终,这些“堤坝”就形成了图像的分割边界。