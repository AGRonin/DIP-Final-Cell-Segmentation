\subsection{U-Net模型讲解} %ybw
U-Net是一种针对生物医学图像分割任务的卷积神经网络架构,由Olaf Ronneberger等人于2015年提出\textsuperscript{\cite{unet}}。如图~\ref{fig:unet}所示,该模型由两部分组成:一个收缩路径用于捕获上下文信息,和一个对称的扩展路径用于精确的局部化。这种结构使得网络能够在有限的训练样本下,提供高精度的分割结果。U-Net的关键设计思想是通过上采样操作恢复高分辨率特征,并与收缩路径中的相应特征图进行拼接,从而提高分割的精确度。

\begin{figure}[htbp]
\centering
\includegraphics[width=\linewidth,trim=0 0 0 0]{./image/unet模型.png}
\bicaption[fig:unet]{图}{\centering U-Net模型原理示意图}{Fig.}{\centering Schematic Diagram of the U-Net Model}
\end{figure}

U-Net的网络结构没有全连接层,而是通过卷积层逐步提升图像的分辨率。在训练过程中,U-Net采用了数据增强技术,这使得网络能够更好地学习到图像中的变形特征,增强其鲁棒性和泛化能力。该架构被广泛应用于多种生物医学图像分割任务,优势在于即使在训练数据较少的情况下,依然能够提供高质量的分割效果,并且具有较快的处理速度。

