\section{项目背景与意义} %本section由zyc修改
细胞图像分割作为医学图像分析的基础环节,直接关系到疾病诊断的准确性、药物筛选的效率和生物学研究的深度。随着数字病理学的快速发展,自动化的细胞分析系统逐渐取代传统的人工显微镜观察,成为现代医学诊断的重要辅助工具。在这一背景下,精准的细胞分割技术不仅能够提高细胞计数的效率,更是实现细胞形态分析、病理分级、疾病预后评估等高级分析任务的先决条件。

在真实的细胞图像中,由于细胞密度高、培养条件限制或组织切片制备等因素,细胞粘连现象普遍存在。以乳腺癌病理切片为例,恶性细胞常呈现密集聚集状态;在血液涂片中,白细胞与红细胞可能发生重叠;在细胞培养实验中,高密度培养会导致细胞边界不清。目前分离细胞的方法只能够分离清晰简单的细胞图像,对于真正出现的情况会存在粘连细胞无法分离、以及过分割问题。这些问题会导致:细胞计数严重偏差、细胞形态参数计算失真、严重影响后续的分类、诊断和定量分析。

当前的研究往往将传统图像处理方法与深度学习方法对立看待,或仅将其作为预处理/后处理步骤简单结合。本研究提出的三级级联架构——Otsu阈值(粗分割)→改进分水岭(解粘连)→U-Net(精细优化)——代表了一种深度融合的创新思路。