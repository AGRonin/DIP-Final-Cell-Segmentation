\subsection{使用U-Net进行前后景分割}

如上文所言,我们在使用简单二值化方法时发现,二值化方法具有阈值不好确定、对于特殊图像效果极差的问题,所以我们考虑引入常用于医学图像分割的U-Net进行前后景分割。

由于这里我们只是引入U-Net卷积神经网络作为一个小小的步骤,而不是整个流程的核心,所以也不算是违背了最好不要使用神经网络简单训练一个模型的要求。

具体来讲,对于训练集,我们以其原图像和结果细胞图(原数据集中每个细胞一张图,这里将其合并为一张图)作为样本和标签,使U-Net具有能够区分每个点是前景(细胞)的概率,然后根据概率给定阈值,高于阈值的则认为其为细胞的组成部分,反之则认为其为背景。

首先,我们测试仅改变二值化方法(简单二值化、U-Net二值化),即U-Net+自适应分水岭方法+对原图使用分水岭。依然考虑我们四张典型图片,分割结果如图~\ref{fig:unet+zi+yuan:totalresult}。

\begin{figure}[htbp]
\centering
\includegraphics[width=\linewidth,trim=0 0 0 0]{./image/U-Net+自适应分水岭+原图总结果.png}
\bicaption[fig:unet+zi+yuan:totalresult]{图}{\centering 改进的基于U-Net做前景标记的自适应分水岭结果}{Fig.}{\centering The Result of Improved Adaptive Watershed with U-Net Foreground Markers}
\end{figure}

首先,从分割细胞数量平均相对误差来看,其结果相较于改进的自适应分水岭(基于原图)结果,在其他变量均不变的情况下,从0.2318提升降低到了0.1746。这是一个十分显著的提升。观察结果图,我们可以看到,其分割结果好了许多,尤其是第三、四图。第三图的左下角部分的光线没有被识别为细胞进行分割。第四图没有被完全整体识别为一个细胞,其环形结构被识别,部分边界被分开。第一、二图的结果也比之前的要更好,一部分连结细胞被分开。但是仍然存一些欠分割问题。

\subsection{使用变值方法基于细胞大小进行注水点选取}

如上文所言,我们的自适应分水岭算法是根据极大值点确定注水点的,其中有一定的要求,比如说min\_distance代表如果两个极大值点的差距小于min\_distance的话就不将其视为两个。那么这里我们发现,这对于不同细胞的适应能力是较差的。有的细胞比较大,那这个数值就应当相应的提高,反之则应当相应变小。所以接下来我们使用变值方法决定min\_distance。

这我们定义,min\_distance是0.8倍的最大距离(后续0.8会进行修改,在最后进行消融实验)。实验结果如图~\ref{fig:unet+zi+bian+yuan:totalresult}。

\begin{figure}[htbp]
\centering
\includegraphics[width=\linewidth,trim=0 0 0 0]{./image/U-Net+自适应分水岭+变值+原图总结果.png}
\bicaption[fig:unet+zi+bian+yuan:totalresult]{图}{\centering 改进的基于U-Net做前景标记的使用变阈值的自适应分水岭结果}{Fig.}{\centering The Result of U-Net-based Adaptive Watershed with Adaptive Thresholds}
\end{figure}

从分割细胞数量平均相对误差来看,其达到0.1721,相比前文有所增长但幅度不大。从结果图来看,其对第二图的结果进行了尤其的改变,分割更多了,解决了一定的欠分割问题。与此同时,其导致了过分割与欠分割并存的问题。

\subsection{使用梯度作为分水岭图像}

我们再来尝试梯度图像。在本章前半部分,在基于二值化的分割方法之中,我们发现梯度对于结果的影响几乎没有。这是真的吗?如果是这样的话为什么教材中会写一般使用梯度图像进行分水岭呢?于是在我们使用U-Net做前景标记之后,我们决定使用梯度图像看一下效果。结果如图~\ref{fig:unet+zi+bian+grad:totalresult}。

\begin{figure}[htbp]
\centering
\includegraphics[width=\linewidth,trim=0 0 0 0]{./image/U-Net+自适应分水岭+变值+梯度总结果.png}
\bicaption[fig:unet+zi+bian+grad:totalresult]{图}{\centering 改进的基于U-Net做前景标记的使用变阈值的使用梯度图像的自适应分水岭结果}{Fig.}{\centering The Result of U-Net-based Adaptive Watershed with Adaptive Thresholds and Gradient Images}
\end{figure}

从分割细胞数量平均相对误差来看,其达到0.1346,相较前文有较大的提升。从结果图上来看,其更清晰的关注到了边界,开始关注内部的线条。但是,其对于概率图变色过于敏感可能导致结果不够好,如第四图出现从中间直接连到边界的不完全分割情况。

\subsection{使用梯度-原始概率图加权图像作为分水岭图像}
既然梯度单独不行,那么我们同时考虑梯度和原始概率图,将二者加权以期待更好的结果,这里的权重为梯度图$0.01$、原始概率图$0.99$。结果如图~\ref{fig:unet+zi+bian+quan:totalresult}。

\begin{figure}[htbp]
\centering
\includegraphics[width=\linewidth,trim=0 0 0 0]{./image/U-Net+自适应分水岭+变值+加权总结果.png}
\bicaption[fig:unet+zi+bian+quan:totalresult]{图}{\centering 改进的基于U-Net做前景标记的使用变阈值的使用梯度-概率加权图像的自适应分水岭结果}{Fig.}{\centering The Result of U-Net-based Adaptive Watershed Using Gradient–Probability Weighted Images}
\end{figure}

从分割细胞数量平均相对误差来看,其达到0.1323,相较前文又有一定的提升。从结果图上来看,其改进了从中间直接连到边界的不完全分割情况,分割结果越来越好。但是,注意第一图中下部分,出现“甜甜圈”现象。即,对于细胞和细胞核,呈现从细胞核到细胞边缘被切成多份的情况。

\subsection{使用形态学方法改善“甜甜圈”问题}

我们引入形态学内容来改进“甜甜圈”现象。当一个分割结果形状接近圆、椭圆且离心率较小的时候,我们认为这已经是一个分割好的细胞了不再需要进行分割。其结果如图~\ref{fig:unet+zi+bian+xing+quan:totalresult}。

\begin{figure}[htbp]
\centering
\includegraphics[width=\linewidth,trim=0 0 0 0]{./image/U-Net+自适应分水岭+变值+形态+加权总结果.png}
\bicaption[fig:unet+zi+bian+xing+quan:totalresult]{图}{\centering 改进的基于U-Net做前景标记的加入形态学判断的使用变阈值的使用梯度-概率加权图像的自适应分水岭结果}{Fig.}{\centering The Result of U-Net-based Adaptive Watershed with Morphology and Weighted Images}
\end{figure}

从分割细胞数量平均相对误差来看,其变差了一点,变为0.1334。但是从结果图上来看,其变好了一些,第一图的“甜甜圈”现象被正确解决。详细分析见本文第5章。