\section{总结} %本section由zyc修改

\subsection{研究总结}
研究初期,我们采用经典的Otsu自适应阈值分割算法作为基础方法。但是在实验中我们发现了他的局限性:Otsu算法仅能进行全局阈值分割,无法区分相互粘连的细胞个体。即当细胞密度较高或培养条件导致细胞紧密接触时,Otsu算法将整个粘连区域识别为单一前景目标,这一缺陷严重影响了细胞计数和形态分析的准确性。

为解决细胞粘连问题,我们引入了分水岭算法这一经典的形态学分割方法,并且在实验中进行分水岭算法的改进。分水岭算法的引入使分割性能得改善,但是也出现了新的问题:对于噪声干扰较大或细胞边界模糊的图像,分水岭算法易产生严重的过分割;而在细胞极度粘连的情况下,算法仍无法准确识别所有细胞个体。

为进一步提升分割精度,特别是改善粘连处理能力,我们将深度学习框架U-Net与改进分水岭算法融合。在这个过程中,我们尝试了形态学、变参数、加权梯度等方法来进行改进。最终在量化指标上实现了从0.2920到0.1334的显著提升。

\subsection{算法的适用条件}
本研究基于2018 Data Science Bowl数据集进行算法开发与优化,在此数据集上的系统性实验表明,本文提出的方法在特定条件下展现出卓越的分割性能。

在高质量图像的分割方面,对于该数据集中图像质量优异、细胞形态清晰、对比度高的样本,本方法能够实现接近完美的分割效果。如图~\ref{fig:result7},在选定的优质测试图像中,本方法的分割结果与基准标注基本一致。
\begin{figure}[htbp]
\centering
\includegraphics[width=\linewidth,trim=0 0 0 0]{./image/结果用图-优质图像.png}
\bicaption[fig:result7]{图}{\centering 改进的基于U-Net做前景标记的使用变阈值的使用梯度-概率加权图像的自适应分水岭结果7}{Fig.}{\centering The Result7 of U-Net-based Adaptive Watershed Using Gradient–Probability Weighted Images}
\end{figure}

值得注意的是,在实际的细胞成像过程中,受限于实验条件、染色技术、显微镜性能等因素,并非所有图像都能达到理想质量。本研究针对这类低质量图像具有一定的分割能力,并且可以看到一些人眼看不到的细胞如图~\ref{fig:result8},这说明我们的算法在较复杂的环境中依然可以维持一定的性能。

\begin{figure}[htbp]
\centering
\includegraphics[width=\linewidth,trim=0 0 0 0]{./image/肉眼不可见细胞.png}
\bicaption[fig:result8]{图}{\centering 改进的基于U-Net做前景标记的使用变阈值的使用梯度-概率加权图像的自适应分水岭结果8}{Fig.}{\centering The Result8 of U-Net-based Adaptive Watershed Using Gradient–Probability Weighted Images}
\end{figure}

由于我们只使用了2018 Data Science Bowl数据集进行算法开发与优化,没有针对其他数据集进行测试,所以不能保证该算法在其他数据集上的适用性。在实际使用过程中,应当结合数据的自身特点对本算法进行调整、测试,以求得更高的性能。


\subsection{改进方向与未来工作}
尽管U-Net的引入显著提升了细胞分割的整体性能,但我们在实验过程中发现了一个新的问题,即空心现象,如图~\ref{fig:空心}。在某些情况下,特别是对于较大或形态不规则的细胞,U-Net输出的分割结果会在细胞内部产生空洞或不连续区域。后续可能可以进一步通过将U-Net与二值化进行结合来改进此问题。更具体的论述详见第5.2节。

另外,本次实验中部分参数选取基于经验值,后续可以进一步考察不同的参数对实验结果的改善。

在生物学实验中,数细胞往往由人工进行。可以说,本实验的结果结合少量人工对个别图像进行再判断将会节省很大的时间以及精力,在实际使用中提供很大的便利。希望在后续改进中,建立更加科学严谨的实验设计和方法评估体系,并不断改进算法,推动该细胞图像分割技术更加实用。
