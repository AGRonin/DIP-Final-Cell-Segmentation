\section{总结} %本section由zyc修改

\subsection{研究总结}
研究初期,我们采用经典的Otsu自适应阈值分割算法作为基础方法。但是在实验中我们发现了他的局限性:Otsu算法仅能进行全局阈值分割,无法区分相互粘连的细胞个体。即当细胞密度较高或培养条件导致细胞紧密接触时,Otsu算法将整个粘连区域识别为单一前景目标,这一缺陷严重影响了细胞计数和形态分析的准确性。

为解决细胞粘连问题,我们引入了分水岭算法这一经典的形态学分割方法,并且在实验中进行分水岭算法的改进。分水岭算法的引入使分割性能得改善,但是也出现了新的问题:对于噪声干扰较大或细胞边界模糊的图像,分水岭算法易产生严重的过分割;而在细胞极度粘连的情况下,算法仍无法准确识别所有细胞个体。

为进一步提升分割精度,特别是改善噪声鲁棒性和重度粘连处理能力,我们将深度学习框架U-Net与传统分水岭算法深度融合。在这个过程中,我们尝试了形态学的方法,以及加权梯度的方法来进行改进。最终在量化指标上实现了从0.2920到0.1334的显著提升。

\subsection{算法的适用条件}
本研究基于2018 Data Science Bowl(DSB2018)细胞核分割数据集进行算法开发与优化,在此数据集上的系统性实验表明,本文提出的方法在特定条件下展现出卓越的分割性能。

在高质量图像的分割方面,对于DSB2018数据集中图像质量优异、细胞形态清晰、对比度高的样本,本方法能够实现接近完美的分割效果。如图~\ref{fig:result},在选定的优质测试图像中,本方法的分割结果与金标准标注基本完全一致,Dice系数达到0.95以上。
\begin{figure}[H]
\centering
\includegraphics[width=\linewidth,trim=0 0 0 0]{./image/结果用图-优质图像.png}
\bicaption[fig:result]{图}{\centering 优质图像分割}{Fig.}{\centering High-quality Segmentation Results}
\end{figure}
值得注意的是,在实际的细胞成像过程中,受限于实验条件、染色技术、显微镜性能等因素,并非所有图像都能达到理想质量。本研究针对这类低质量图像具有一定的分割能力,并且可以看到一些人眼看不到的细胞如图~\ref{fig:result6},这是非常有优势的。
\begin{figure}[H]
\centering
\includegraphics[width=\linewidth,trim=0 0 0 0]{./image/肉眼不可见细胞.png}
\bicaption[fig:result6]{图}{\centering 结果图}{Fig.}{\centering Result Image}
\end{figure}


\subsection{改进方向与未来工作}
尽管U-Net的引入显著提升了细胞分割的整体性能,但我们在实验过程中发现了一个新的问题——空心化现象(Cell Hollowing Effect)如图~\ref{fig:result7}。在某些情况下,特别是对于较大或形态不规则的细胞,U-Net输出的分割结果会在细胞内部产生空洞或不连续区域。后续我们可以进一步通过将U-Net与二值化进行结合来改进此问题。本次实验将两者进行结合时,并未带来很好的效果。但是考虑本次结合的参数选取大多基于经验,后续可以进一步考察不同的参数对实验结果的改善。
\begin{figure}[H]
\centering
\includegraphics[width=\linewidth,trim=0 0 0 0]{./image/空心.png}
\bicaption[fig:result7]{图}{\centering 空心现象}{Fig.}{\centering Hollow Artifact Example}
\end{figure}

本实验的结果结合人工对个别图像进行判断将会节省很大的时间以及精力,在实际使用中提供很大的便利。本研究在后续改进中,需要建立更加科学严谨的实验设计和方法评估体系,推动细胞图像分割技术从实验室研究向临床实用稳步迈进。
