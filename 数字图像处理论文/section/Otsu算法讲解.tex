\subsection{Otsu算法讲解} %jhr
Otsu算法是一类基于图像灰度直方图的自动阈值分割方法。其基本思想是通过最大化类间方差来确定最佳阈值,从而将图像分割为前景和背景两部分。Otsu算法的步骤如下:
\begin{enumerate}[itemsep=0pt,topsep=2pt]
    \item 计算输入图像的归一化直方图。使用$p_i$表示该直方图的各个分量。
    \item 用下式,对$k$ ($k=0,1,...,L-1$)($L$为灰度级数),计算累计和$p_1(k)$。
    \begin{equation}
        p_1(k) = \sum_{i=0}^{k} p_i
    \end{equation}
    \item 用下式,计算累计均值$m(k)$。
    \begin{equation}
        m(k) = \sum_{i=0}^{k} i \cdot p_i
    \end{equation}
    \item 计算全局灰度均值$m_G$。
    \begin{equation}
        m_G = \sum_{i=0}^{L-1} i \cdot p_i
    \end{equation}
    \item 对于$k$ ($k=0,1,...,L-1$),计算类间方差$\sigma^2_B(k)$。
    \begin{equation}
        \sigma^2_B(k) = \frac{[m_G \cdot p_1(k) - m(k)]^2}{p_1(k) \cdot [1 - p_1(k)]}
    \end{equation}
    \item 得到最大类间方差对应的$k$值,即为最佳阈值$k^*$。
    \begin{equation}
        \sigma_B^2(k^*) = \max_{0 \le k \le L-1} \sigma_B^2(k)
    \end{equation}

\end{enumerate}