\section{实验数据说明} %本section由zyc修改
\subsection{数据集来源与选择依据}

本研究采用2018 Data Science Bowl 细胞核分割挑战赛的公开数据集作为实验数据。该数据集数据质量高:图像由专业生物学家标注,分割标签精确可靠;多样性丰富:包含多种细胞类型、不同染色方法、不同显微镜获取的图像;挑战性适中:包含不同粘连程度的细胞,适合验证本方法的有效性;社区认可度高:被广泛用作细胞分割的基准数据集,便于结果对比。

\subsection{数据集使用说明}
在DSB2018数据集中,我们仅使用stage1\_train子集,图像尺寸在$256\times256$至$1024\times1024$像素之间,涵盖了多种细胞类型和染色方法,数据集中细胞粘连程度分布均匀,包含轻度、中度和重度粘连样本。在实验的过程中,我们将所有的图像尺寸都转化成了$256\times256$。

为确保实验的可靠性与结果的可比性,我们将数据集按$70\%:15\%:15\%$的比例划分为训练集、验证集和测试集。

\subsection{四张典型图像选择说明}

我们选择数据集中的四幅图像作为本文中的主要展示图像。选择典型图像的考虑因素有:细胞类型、细胞分布、可能的分割错误风险、易于对比。

\subsubsection{典型图像1}
如~\ref{fig:c1}所示,细胞形状近似圆形或椭圆形,细胞分布间距适中且轻度粘连,光照均匀,噪声较少。

\begin{figure}[htbp]
\centering
\includegraphics[width=\linewidth,trim=0 0 0 0]{./image/分割_原图.jpg}
\bicaption[fig:c1]{图}{\centering 典型图像一:分割}{Fig.}{\centering Typical Image 1: Segmentation Structure}
\end{figure}

\subsubsection{典型图像2}
如~\ref{fig:c2}所示,细胞形状细长形放射状排列,细胞分布中心紧密向外辐射,内部边界模糊但外部边界清晰。

\begin{figure}[htbp]
\centering
\includegraphics[width=\linewidth,trim=0 0 0 0]{./image/车轮_原图.png}
\bicaption[fig:c2]{图}{\centering 典型图像二:车轮}{Fig.}{\centering Typical Image 2: Wheel Structure}
\end{figure}

\subsubsection{典型图像3}
如~\ref{fig:c3}所示,细胞形状不规则,左下角有不均匀光照。
\begin{figure}[htbp]
\centering
\includegraphics[width=\linewidth,trim=0 0 0 0]{./image/光_原图.png}
\bicaption[fig:c3]{图}{\centering 典型图像三:光}{Fig.}{\centering Typical Image 3: Light Structure}
\end{figure}

\subsubsection{典型图像4}
如~\ref{fig:c4}所示,细胞形状不规则,有环形细胞(可能是中心细胞核,与背景颜色一致)。
\begin{figure}[htbp]
\centering
\includegraphics[width=\linewidth,trim=0 0 0 0]{./image/甜甜圈_原图.jpg}
\bicaption[fig:c4]{图}{\centering 典型图像四:甜甜圈}{Fig.}{\centering Typical Image 4: Doughnut Structure}
\end{figure}

\subsection{GT Mask图}

本实验以数据集中给出结果图~\ref{fig:mask}作为正确结果,将这些结果图对齐合并,用来核验我们代码得到的分割结果并作为数值结果的参考依据。

\begin{figure}[htbp]
\centering
\includegraphics[width=\linewidth,trim=0 0 0 0]{./image/GT-Mask图.png}
\bicaption[fig:mask]{图}{\centering GT Mask图}{Fig.}{\centering Image : GT Mask}
\end{figure}